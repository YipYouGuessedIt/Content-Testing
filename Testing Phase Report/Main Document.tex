\documentclass[12pt, oneside]{article}
 
\usepackage{graphicx}
\usepackage{hyperref}
\graphicspath{ {Images/} }

\begin{document}
\thispagestyle{empty}

\normalfont
\begin{center}
{\fontsize{2cm}{2em}\selectfont Content Testing  \\}
\vspace{1cm}
\begin{minipage}{0.75\linewidth}
    \centering

    {\uppercase{\Large COS 301 Assignment\par}}
   	{\uppercase{\Large Content Testing Team \par}}
    \vspace{1cm}

    {\normalsize Duran Cole (13329414)\par}
    {\normalsize Kyhle Ohlinger (11131952)\par}
    {\normalsize Andreas du Preez (12207871)\par}
    {\normalsize Aluwani Simetsi (11322935)\par}
    {\normalsize Michelle Swanepoel (13066294)\par}
    {\normalsize Zander Boshoff (12035671)\par}
    {\normalsize Tebogo Seshibe (13181442)\par}
    {\normalsize Kgomotso Sito (12243273)\par}
    {\normalsize Thabang Letageng (13057937)\par}
    \vspace{1cm}
    
    {\Large April 2015\par}
    \vspace{1cm}
    {\Large Github Links \par}
    \href{https://github.com/YipYouGuessedIt/Content-Testing}{Content Testing}\linebreak
    \href{https://github.com/masterfloki/cos301_Phase3_Content}{Team A files}\linebreak
    \href{https://github.com/BuzzSpaceB/Content}{Team B files}
    \vspace{1cm}\linebreak
    {\large version: 1.0 }
\end{minipage}
\end{center}
\clearpage


\pagenumbering{roman}
\tableofcontents
\newpage
\pagenumbering{arabic}

\newpage
\section{Introduction} 
	\paragraph{}
	The purpose of this document is to provide incite into what was produced by both Content  Team A (Buzz++) and Content Team B (D3). Groups were to provide an infrastructure to their low level teams and then build integration tests that would then allow the functional teams produced code to be tested and eventually added into the system. Tests were done per sub module (e.g. reporting, resource, threads and status) as to break up the integration that was to be done and provide and easier way to compare the teams work and functional code.
	
\section{Threads Integration}
The tests provided below show all cases in which the Threads module was to make use of another Content module and the expected and actual results of each tests. We then explain which non-functional requirements have not been fulfilled in the overall Threads module.
	\subsection{Threads Team A}	
		\subsubsection{Functional Testing}
			if success:
\paragraph{Test One Name}
	\subparagraph{Expected Test output}
	Please write what the client expected the test to do.
	
	\subparagraph{Actual Test output}
	This test was a success 
		
if it is not a success:
\paragraph{Test Two Name}
	\subparagraph{Expected Test output}
	Please write what the client expected the test to do.
	
	\subparagraph{Actual Test output}
	This test was a failure for the following reasons
		\begin{itemize}
			\item because the team didn't implement this
			\item because testing
		\end{itemize}
		\subsubsection{Non-Functional Testing }
			\section{Quality Requirements}
\paragraph{This section looks st the manner in which the code is made and maintained}
\section{Maintainability}
	\begin{itemize}
		\item How flexible and extensible is the code and, can the system be easily maintained in future?
		\item The developers implemented their system very simplistically and provided comments with explanations of their code. The comments provided contain expected parameters, expected results and, potential throws.
	\end{itemize}
\section{Scalability}
	\begin{itemize}
		\item The system should be easily scalable to large systems from the University of Pretoria to even larger systems.
		\item Due to it's simplistic nature and implementation, the system could easily be scaled.
	\end{itemize}
\section{Performance Requirements}
	\begin{itemize}
		\item Reporting operations should take no more than 5 seconds while non-reporting operations should take less than (or equal to) 0.2 seconds.
		\item This system does not work so we cannot test.
	\end{itemize}
\section{Reliability and Availability}
	\begin{itemize}
		\item Should support fail-safe over safety of components and deployment without a single point of failure.
		\item The system itself completely failed as there were multiple errors.
	\end{itemize}
\section{Security}
	\begin{itemize}
		\item Check authentication against LDAP and flexible configurable authorization framework.
		\item No authorization is handled.
	\end{itemize}
\section{Auditability}
	\begin{itemize}
		\item The systems logs all requests and responses as stringified JSON object.
		\item They printed out whenever a function was called but not kept in a database.
	\end{itemize}
\section{Testability}
	\begin{itemize}
		\item Must be testable through unit testing and integration tests. Should verify that pre and post conditions met.
		\item Unit tests have been implemented but they do not succeed as the system itself is flawed.
	\end{itemize}
\section{Usability}
	\begin{itemize}
		\item The system should be intuitive and fun to use.
		\item This section is directed more towards the final end-product.
	\end{itemize}
\section{Integrability}
	\begin{itemize}
		\item The system should be able to easily integrate future requirements.
		\item The simplistic nature allows for this.
	\end{itemize}
\section{Deployability}
	\begin{itemize}
		\item The system must be deployable on Linux servers, different database persistence environments and, diferent repositories for authentication credentials.
		\item The system is deployable on Linux server.
	\end{itemize}

	
	\subsection{Threads Team B}	
		\subsubsection{Functional Testing}
			if success:
\paragraph{Test One Name}
	\subparagraph{Expected Test output}
	Please write what the client expected the test to do.
	
	\subparagraph{Actual Test output}
	This test was a success 
		
if it is not a success:
\paragraph{Test Two Name}
	\subparagraph{Expected Test output}
	Please write what the client expected the test to do.
	
	\subparagraph{Actual Test output}
	This test was a failure for the following reasons
		\begin{itemize}
			\item because the team didn't implement this
			\item because testing
		\end{itemize}
		\subsubsection{Non-Functional Testing }
			\section{Quality Requirements}
\paragraph{This section looks st the manner in which the code is made and maintained}
\section{Maintainability}
	\begin{itemize}
		\item How flexible and extensible is the code and, can the system be easily maintained in future?
		\item The developers implemented their system very simplistically and provided comments with explanations of their code. The comments provided contain expected parameters, expected results and, potential throws.
	\end{itemize}
\section{Scalability}
	\begin{itemize}
		\item The system should be easily scalable to large systems from the University of Pretoria to even larger systems.
		\item Due to it's simplistic nature and implementation, the system could easily be scaled.
	\end{itemize}
\section{Performance Requirements}
	\begin{itemize}
		\item Reporting operations should take no more than 5 seconds while non-reporting operations should take less than (or equal to) 0.2 seconds.
		\item This system does not work so we cannot test.
	\end{itemize}
\section{Reliability and Availability}
	\begin{itemize}
		\item Should support fail-safe over safety of components and deployment without a single point of failure.
		\item The system itself completely failed as there were multiple errors.
	\end{itemize}
\section{Security}
	\begin{itemize}
		\item Check authentication against LDAP and flexible configurable authorization framework.
		\item No authorization is handled.
	\end{itemize}
\section{Auditability}
	\begin{itemize}
		\item The systems logs all requests and responses as stringified JSON object.
		\item They printed out whenever a function was called but not kept in a database.
	\end{itemize}
\section{Testability}
	\begin{itemize}
		\item Must be testable through unit testing and integration tests. Should verify that pre and post conditions met.
		\item Unit tests have been implemented but they do not succeed as the system itself is flawed.
	\end{itemize}
\section{Usability}
	\begin{itemize}
		\item The system should be intuitive and fun to use.
		\item This section is directed more towards the final end-product.
	\end{itemize}
\section{Integrability}
	\begin{itemize}
		\item The system should be able to easily integrate future requirements.
		\item The simplistic nature allows for this.
	\end{itemize}
\section{Deployability}
	\begin{itemize}
		\item The system must be deployable on Linux servers, different database persistence environments and, diferent repositories for authentication credentials.
		\item The system is deployable on Linux server.
	\end{itemize}

	
	\subsection{Threads Comparison}
	According to the above tests both teams failed to integrate Threads with in the overall system, as all covered tests were failures and did not provide the expected functionality for the reasons given above.
	
\newpage 
\section{Status Integration}
The tests provided below show all cases in which the Status module was to make use of another Content module and the expected and actual results of each tests. We then explain which non-functional requirements have not been fulfilled in the overall Status module.
	\subsection{Status Team A}	
		\subsubsection{Functional Testing}
			if success:
\paragraph{Test One Name}
	\subparagraph{Expected Test output}
	Please write what the client expected the test to do.
	
	\subparagraph{Actual Test output}
	This test was a success 
		
if it is not a success:
\paragraph{Test Two Name}
	\subparagraph{Expected Test output}
	Please write what the client expected the test to do.
	
	\subparagraph{Actual Test output}
	This test was a failure for the following reasons
		\begin{itemize}
			\item because the team didn't implement this
			\item because testing
		\end{itemize}
		\subsubsection{Non-Functional Testing }
			\section{Quality Requirements}
\paragraph{This section looks st the manner in which the code is made and maintained}
\section{Maintainability}
	\begin{itemize}
		\item How flexible and extensible is the code and, can the system be easily maintained in future?
		\item The developers implemented their system very simplistically and provided comments with explanations of their code. The comments provided contain expected parameters, expected results and, potential throws.
	\end{itemize}
\section{Scalability}
	\begin{itemize}
		\item The system should be easily scalable to large systems from the University of Pretoria to even larger systems.
		\item Due to it's simplistic nature and implementation, the system could easily be scaled.
	\end{itemize}
\section{Performance Requirements}
	\begin{itemize}
		\item Reporting operations should take no more than 5 seconds while non-reporting operations should take less than (or equal to) 0.2 seconds.
		\item This system does not work so we cannot test.
	\end{itemize}
\section{Reliability and Availability}
	\begin{itemize}
		\item Should support fail-safe over safety of components and deployment without a single point of failure.
		\item The system itself completely failed as there were multiple errors.
	\end{itemize}
\section{Security}
	\begin{itemize}
		\item Check authentication against LDAP and flexible configurable authorization framework.
		\item No authorization is handled.
	\end{itemize}
\section{Auditability}
	\begin{itemize}
		\item The systems logs all requests and responses as stringified JSON object.
		\item They printed out whenever a function was called but not kept in a database.
	\end{itemize}
\section{Testability}
	\begin{itemize}
		\item Must be testable through unit testing and integration tests. Should verify that pre and post conditions met.
		\item Unit tests have been implemented but they do not succeed as the system itself is flawed.
	\end{itemize}
\section{Usability}
	\begin{itemize}
		\item The system should be intuitive and fun to use.
		\item This section is directed more towards the final end-product.
	\end{itemize}
\section{Integrability}
	\begin{itemize}
		\item The system should be able to easily integrate future requirements.
		\item The simplistic nature allows for this.
	\end{itemize}
\section{Deployability}
	\begin{itemize}
		\item The system must be deployable on Linux servers, different database persistence environments and, diferent repositories for authentication credentials.
		\item The system is deployable on Linux server.
	\end{itemize}

	
	\subsection{Status Team B}	
		\subsubsection{Functional Testing}
			if success:
\paragraph{Test One Name}
	\subparagraph{Expected Test output}
	Please write what the client expected the test to do.
	
	\subparagraph{Actual Test output}
	This test was a success 
		
if it is not a success:
\paragraph{Test Two Name}
	\subparagraph{Expected Test output}
	Please write what the client expected the test to do.
	
	\subparagraph{Actual Test output}
	This test was a failure for the following reasons
		\begin{itemize}
			\item because the team didn't implement this
			\item because testing
		\end{itemize}
		\subsubsection{Non-Functional Testing }
			\section{Quality Requirements}
\paragraph{This section looks st the manner in which the code is made and maintained}
\section{Maintainability}
	\begin{itemize}
		\item How flexible and extensible is the code and, can the system be easily maintained in future?
		\item The developers implemented their system very simplistically and provided comments with explanations of their code. The comments provided contain expected parameters, expected results and, potential throws.
	\end{itemize}
\section{Scalability}
	\begin{itemize}
		\item The system should be easily scalable to large systems from the University of Pretoria to even larger systems.
		\item Due to it's simplistic nature and implementation, the system could easily be scaled.
	\end{itemize}
\section{Performance Requirements}
	\begin{itemize}
		\item Reporting operations should take no more than 5 seconds while non-reporting operations should take less than (or equal to) 0.2 seconds.
		\item This system does not work so we cannot test.
	\end{itemize}
\section{Reliability and Availability}
	\begin{itemize}
		\item Should support fail-safe over safety of components and deployment without a single point of failure.
		\item The system itself completely failed as there were multiple errors.
	\end{itemize}
\section{Security}
	\begin{itemize}
		\item Check authentication against LDAP and flexible configurable authorization framework.
		\item No authorization is handled.
	\end{itemize}
\section{Auditability}
	\begin{itemize}
		\item The systems logs all requests and responses as stringified JSON object.
		\item They printed out whenever a function was called but not kept in a database.
	\end{itemize}
\section{Testability}
	\begin{itemize}
		\item Must be testable through unit testing and integration tests. Should verify that pre and post conditions met.
		\item Unit tests have been implemented but they do not succeed as the system itself is flawed.
	\end{itemize}
\section{Usability}
	\begin{itemize}
		\item The system should be intuitive and fun to use.
		\item This section is directed more towards the final end-product.
	\end{itemize}
\section{Integrability}
	\begin{itemize}
		\item The system should be able to easily integrate future requirements.
		\item The simplistic nature allows for this.
	\end{itemize}
\section{Deployability}
	\begin{itemize}
		\item The system must be deployable on Linux servers, different database persistence environments and, diferent repositories for authentication credentials.
		\item The system is deployable on Linux server.
	\end{itemize}

	
	\subsection{Status Comparison}
	Team A managed to implement 100 percent of the functions they had to do where as Team B only managed to implement a third of the  functionality. Overall however both tests failed in a sense because of the failure to run on Linux that Team A presented and the lack of functionality by Team B.
	
\newpage 
\section{Resources Integration}
The tests provided below show all cases in which the Resources module was to make use of another Content module and the expected and actual results of each tests. We then explain which non-functional requirements have not been fulfilled in the overall Resources module.
	\subsection{Resources Team A}	
		\subsubsection{Functional Testing}
			if success:
\paragraph{Test One Name}
	\subparagraph{Expected Test output}
	Please write what the client expected the test to do.
	
	\subparagraph{Actual Test output}
	This test was a success 
		
if it is not a success:
\paragraph{Test Two Name}
	\subparagraph{Expected Test output}
	Please write what the client expected the test to do.
	
	\subparagraph{Actual Test output}
	This test was a failure for the following reasons
		\begin{itemize}
			\item because the team didn't implement this
			\item because testing
		\end{itemize}
		\subsubsection{Non-Functional Testing }
			\section{Quality Requirements}
\paragraph{This section looks st the manner in which the code is made and maintained}
\section{Maintainability}
	\begin{itemize}
		\item How flexible and extensible is the code and, can the system be easily maintained in future?
		\item The developers implemented their system very simplistically and provided comments with explanations of their code. The comments provided contain expected parameters, expected results and, potential throws.
	\end{itemize}
\section{Scalability}
	\begin{itemize}
		\item The system should be easily scalable to large systems from the University of Pretoria to even larger systems.
		\item Due to it's simplistic nature and implementation, the system could easily be scaled.
	\end{itemize}
\section{Performance Requirements}
	\begin{itemize}
		\item Reporting operations should take no more than 5 seconds while non-reporting operations should take less than (or equal to) 0.2 seconds.
		\item This system does not work so we cannot test.
	\end{itemize}
\section{Reliability and Availability}
	\begin{itemize}
		\item Should support fail-safe over safety of components and deployment without a single point of failure.
		\item The system itself completely failed as there were multiple errors.
	\end{itemize}
\section{Security}
	\begin{itemize}
		\item Check authentication against LDAP and flexible configurable authorization framework.
		\item No authorization is handled.
	\end{itemize}
\section{Auditability}
	\begin{itemize}
		\item The systems logs all requests and responses as stringified JSON object.
		\item They printed out whenever a function was called but not kept in a database.
	\end{itemize}
\section{Testability}
	\begin{itemize}
		\item Must be testable through unit testing and integration tests. Should verify that pre and post conditions met.
		\item Unit tests have been implemented but they do not succeed as the system itself is flawed.
	\end{itemize}
\section{Usability}
	\begin{itemize}
		\item The system should be intuitive and fun to use.
		\item This section is directed more towards the final end-product.
	\end{itemize}
\section{Integrability}
	\begin{itemize}
		\item The system should be able to easily integrate future requirements.
		\item The simplistic nature allows for this.
	\end{itemize}
\section{Deployability}
	\begin{itemize}
		\item The system must be deployable on Linux servers, different database persistence environments and, diferent repositories for authentication credentials.
		\item The system is deployable on Linux server.
	\end{itemize}

	
	\subsection{Resources Team B}	
		\subsubsection{Functional Testing}
			if success:
\paragraph{Test One Name}
	\subparagraph{Expected Test output}
	Please write what the client expected the test to do.
	
	\subparagraph{Actual Test output}
	This test was a success 
		
if it is not a success:
\paragraph{Test Two Name}
	\subparagraph{Expected Test output}
	Please write what the client expected the test to do.
	
	\subparagraph{Actual Test output}
	This test was a failure for the following reasons
		\begin{itemize}
			\item because the team didn't implement this
			\item because testing
		\end{itemize}
		\subsubsection{Non-Functional Testing }
			\section{Quality Requirements}
\paragraph{This section looks st the manner in which the code is made and maintained}
\section{Maintainability}
	\begin{itemize}
		\item How flexible and extensible is the code and, can the system be easily maintained in future?
		\item The developers implemented their system very simplistically and provided comments with explanations of their code. The comments provided contain expected parameters, expected results and, potential throws.
	\end{itemize}
\section{Scalability}
	\begin{itemize}
		\item The system should be easily scalable to large systems from the University of Pretoria to even larger systems.
		\item Due to it's simplistic nature and implementation, the system could easily be scaled.
	\end{itemize}
\section{Performance Requirements}
	\begin{itemize}
		\item Reporting operations should take no more than 5 seconds while non-reporting operations should take less than (or equal to) 0.2 seconds.
		\item This system does not work so we cannot test.
	\end{itemize}
\section{Reliability and Availability}
	\begin{itemize}
		\item Should support fail-safe over safety of components and deployment without a single point of failure.
		\item The system itself completely failed as there were multiple errors.
	\end{itemize}
\section{Security}
	\begin{itemize}
		\item Check authentication against LDAP and flexible configurable authorization framework.
		\item No authorization is handled.
	\end{itemize}
\section{Auditability}
	\begin{itemize}
		\item The systems logs all requests and responses as stringified JSON object.
		\item They printed out whenever a function was called but not kept in a database.
	\end{itemize}
\section{Testability}
	\begin{itemize}
		\item Must be testable through unit testing and integration tests. Should verify that pre and post conditions met.
		\item Unit tests have been implemented but they do not succeed as the system itself is flawed.
	\end{itemize}
\section{Usability}
	\begin{itemize}
		\item The system should be intuitive and fun to use.
		\item This section is directed more towards the final end-product.
	\end{itemize}
\section{Integrability}
	\begin{itemize}
		\item The system should be able to easily integrate future requirements.
		\item The simplistic nature allows for this.
	\end{itemize}
\section{Deployability}
	\begin{itemize}
		\item The system must be deployable on Linux servers, different database persistence environments and, diferent repositories for authentication credentials.
		\item The system is deployable on Linux server.
	\end{itemize}

			
	
	\subsection{Resources Comparison}
	
	
\newpage 
\section{Reporting Integration}
The tests provided below show all cases in which the Reporting module was to make use of another Content module and the expected and actual results of each tests. We then explain which non-functional requirements have not been fulfilled in the overall Reporting module.
	\subsection{Reporting Team A}	
		\subsubsection{Functional Testing}
			if success:
\paragraph{Test One Name}
	\subparagraph{Expected Test output}
	Please write what the client expected the test to do.
	
	\subparagraph{Actual Test output}
	This test was a success 
		
if it is not a success:
\paragraph{Test Two Name}
	\subparagraph{Expected Test output}
	Please write what the client expected the test to do.
	
	\subparagraph{Actual Test output}
	This test was a failure for the following reasons
		\begin{itemize}
			\item because the team didn't implement this
			\item because testing
		\end{itemize}
		\subsubsection{Non-Functional Testing }
			\section{Quality Requirements}
\paragraph{This section looks st the manner in which the code is made and maintained}
\section{Maintainability}
	\begin{itemize}
		\item How flexible and extensible is the code and, can the system be easily maintained in future?
		\item The developers implemented their system very simplistically and provided comments with explanations of their code. The comments provided contain expected parameters, expected results and, potential throws.
	\end{itemize}
\section{Scalability}
	\begin{itemize}
		\item The system should be easily scalable to large systems from the University of Pretoria to even larger systems.
		\item Due to it's simplistic nature and implementation, the system could easily be scaled.
	\end{itemize}
\section{Performance Requirements}
	\begin{itemize}
		\item Reporting operations should take no more than 5 seconds while non-reporting operations should take less than (or equal to) 0.2 seconds.
		\item This system does not work so we cannot test.
	\end{itemize}
\section{Reliability and Availability}
	\begin{itemize}
		\item Should support fail-safe over safety of components and deployment without a single point of failure.
		\item The system itself completely failed as there were multiple errors.
	\end{itemize}
\section{Security}
	\begin{itemize}
		\item Check authentication against LDAP and flexible configurable authorization framework.
		\item No authorization is handled.
	\end{itemize}
\section{Auditability}
	\begin{itemize}
		\item The systems logs all requests and responses as stringified JSON object.
		\item They printed out whenever a function was called but not kept in a database.
	\end{itemize}
\section{Testability}
	\begin{itemize}
		\item Must be testable through unit testing and integration tests. Should verify that pre and post conditions met.
		\item Unit tests have been implemented but they do not succeed as the system itself is flawed.
	\end{itemize}
\section{Usability}
	\begin{itemize}
		\item The system should be intuitive and fun to use.
		\item This section is directed more towards the final end-product.
	\end{itemize}
\section{Integrability}
	\begin{itemize}
		\item The system should be able to easily integrate future requirements.
		\item The simplistic nature allows for this.
	\end{itemize}
\section{Deployability}
	\begin{itemize}
		\item The system must be deployable on Linux servers, different database persistence environments and, diferent repositories for authentication credentials.
		\item The system is deployable on Linux server.
	\end{itemize}

	
	\subsection{Reporting Team B}	
		\subsubsection{Functional Testing}
			if success:
\paragraph{Test One Name}
	\subparagraph{Expected Test output}
	Please write what the client expected the test to do.
	
	\subparagraph{Actual Test output}
	This test was a success 
		
if it is not a success:
\paragraph{Test Two Name}
	\subparagraph{Expected Test output}
	Please write what the client expected the test to do.
	
	\subparagraph{Actual Test output}
	This test was a failure for the following reasons
		\begin{itemize}
			\item because the team didn't implement this
			\item because testing
		\end{itemize}
		\subsubsection{Non-Functional Testing }
			\section{Quality Requirements}
\paragraph{This section looks st the manner in which the code is made and maintained}
\section{Maintainability}
	\begin{itemize}
		\item How flexible and extensible is the code and, can the system be easily maintained in future?
		\item The developers implemented their system very simplistically and provided comments with explanations of their code. The comments provided contain expected parameters, expected results and, potential throws.
	\end{itemize}
\section{Scalability}
	\begin{itemize}
		\item The system should be easily scalable to large systems from the University of Pretoria to even larger systems.
		\item Due to it's simplistic nature and implementation, the system could easily be scaled.
	\end{itemize}
\section{Performance Requirements}
	\begin{itemize}
		\item Reporting operations should take no more than 5 seconds while non-reporting operations should take less than (or equal to) 0.2 seconds.
		\item This system does not work so we cannot test.
	\end{itemize}
\section{Reliability and Availability}
	\begin{itemize}
		\item Should support fail-safe over safety of components and deployment without a single point of failure.
		\item The system itself completely failed as there were multiple errors.
	\end{itemize}
\section{Security}
	\begin{itemize}
		\item Check authentication against LDAP and flexible configurable authorization framework.
		\item No authorization is handled.
	\end{itemize}
\section{Auditability}
	\begin{itemize}
		\item The systems logs all requests and responses as stringified JSON object.
		\item They printed out whenever a function was called but not kept in a database.
	\end{itemize}
\section{Testability}
	\begin{itemize}
		\item Must be testable through unit testing and integration tests. Should verify that pre and post conditions met.
		\item Unit tests have been implemented but they do not succeed as the system itself is flawed.
	\end{itemize}
\section{Usability}
	\begin{itemize}
		\item The system should be intuitive and fun to use.
		\item This section is directed more towards the final end-product.
	\end{itemize}
\section{Integrability}
	\begin{itemize}
		\item The system should be able to easily integrate future requirements.
		\item The simplistic nature allows for this.
	\end{itemize}
\section{Deployability}
	\begin{itemize}
		\item The system must be deployable on Linux servers, different database persistence environments and, diferent repositories for authentication credentials.
		\item The system is deployable on Linux server.
	\end{itemize}

	
	\subsection{Reporting Comparison}

\end{document}
