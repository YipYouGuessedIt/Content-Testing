\documentclass[12pt, oneside]{article}
 
\usepackage{graphicx}
\usepackage{hyperref}
\graphicspath{ {Images/} }

\begin{document}
\thispagestyle{empty}

\normalfont
\begin{center}
{\fontsize{2cm}{2em}\selectfont Content Testing  \\}
\vspace{1cm}
\begin{minipage}{0.75\linewidth}
    \centering

    {\uppercase{\Large COS 301 Assignment\par}}
   	{\uppercase{\Large Content Testing Team \par}}
    \vspace{1cm}

    {\normalsize Duran Cole (13329414)\par}
    {\normalsize Kyhle Ohlinger (11131952)\par}
    {\normalsize Andreas du Preez (12207871)\par}
    {\normalsize Aluwani Simetsi (11322935)\par}
    {\normalsize Michelle Swanepoel (13066294)\par}
    {\normalsize Zander Boshoff (12035671)\par}
    {\normalsize Tebogo Seshibe (13181442)\par}
    {\normalsize Kgomotso Sito (12243273)\par}
    {\normalsize Thabang Letageng (13057937)\par}
    \vspace{1cm}
    
    {\Large April 2015\par}
    \vspace{1cm}
    {\Large Github Links \par}
    \href{https://github.com/YipYouGuessedIt/Content-Testing}{Content Testing}\linebreak
    \href{https://github.com/masterfloki/cos301_Phase3_Content}{Team A files}\linebreak
    \href{https://github.com/BuzzSpaceB/Content}{Team B files}
    \vspace{1cm}\linebreak
    {\large version: 1.0 }
\end{minipage}
\end{center}
\clearpage


\pagenumbering{roman}
\tableofcontents
\newpage
\pagenumbering{arabic}

\newpage
\section{Introduction} 
	\paragraph{}
	The purpose of this document is to provide incite into what was produced by both Content  Team A (Buzz++) and Content Team B (D3). Groups were to provide an infrastructure to their low level teams and then build integration tests that would then allow the functional teams produced code to be tested and eventually added into the system. Tests were done per sub module (e.g. reporting, resource, threads and status) as to break up the integration that was to be done and provide and easier way to compare the teams work and functional code.
	
\section{Threads Integration}
The tests provided below show all cases in which the Threads module was to make use of another Content module and the expected and actual results of each tests. We then explain which non-functional requirements have not been fulfilled in the overall Threads module.
	\subsection{Threads Team A}	
		\subsubsection{Functional Testing}
			
\paragraph{Requirement 1 - Upload Resources}
	\subparagraph{Expected Test output}
	Add the uploaded resource to the list of resources, and display the list
	
	\subparagraph{Actual Test output}
	Option to upload available, but not functioning (does not add resource to the list) 
		\begin{itemize}
			\item Because the team didn't integrate this, implemented the function
		\end{itemize}
		
\paragraph{Requirement 2 - Remove Resource}
	\subparagraph{Expected Test output}
	Remove resource and display list of resources with resource of interest removed
	
	\subparagraph{Actual Test output}
	Removed resource and displayed list of resources with resource of interest removed
	
\paragraph{Requirement 3 - Add Constraint}
	\subparagraph{Expected Test output}
	Add constraint with specified details and display list of constraints with the constraint of added 
	
	\subparagraph{Actual Test output}
	Added constraint with specified details and display list of constraints with the constraints of added included
	
\paragraph{Requirement 4 - Remove Constraint}
	\subparagraph{Expected Test output}
	Remove constraint and display list of constraints with constraint of interest removed
	
	\subparagraph{Actual Test output}
	Removed constraint and displayed list of constraints with constraint of interest removed
	
\paragraph{Requirement 5 - Update Constraint}
	\subparagraph{Expected Test output}
	Change constraint details to those specified in the field (size) and display list of constraints with the updated constraint included
	
	\subparagraph{Actual Test output}
	Changed constraint details to those specified in the field (size) and displayed list of constraints with the updated constraint included
		\subsubsection{Non-Functional Testing }
			
\paragraph{Security}
	The authorisation in terms of who can change and create calculation of status methods was never done. 
	\subparagraph{Example of problem}
		For example, if a person wanted to change the way a status is calculated, every member of a Buzz Space will be able to do it. Only the lecturers are supposed to be able to do this. 


	
	\subsection{Threads Team B}	
		\subsubsection{Functional Testing}
			
\paragraph{Requirement 1 - Upload Resources}
	\subparagraph{Expected Test output}
	Add the uploaded resource to the list of resources, and display the list
	
	\subparagraph{Actual Test output}
	Option to upload available, but not functioning (does not add resource to the list) 
		\begin{itemize}
			\item Because the team didn't integrate this, implemented the function
		\end{itemize}
		
\paragraph{Requirement 2 - Remove Resource}
	\subparagraph{Expected Test output}
	Remove resource and display list of resources with resource of interest removed
	
	\subparagraph{Actual Test output}
	Removed resource and displayed list of resources with resource of interest removed
	
\paragraph{Requirement 3 - Add Constraint}
	\subparagraph{Expected Test output}
	Add constraint with specified details and display list of constraints with the constraint of added 
	
	\subparagraph{Actual Test output}
	Added constraint with specified details and display list of constraints with the constraints of added included
	
\paragraph{Requirement 4 - Remove Constraint}
	\subparagraph{Expected Test output}
	Remove constraint and display list of constraints with constraint of interest removed
	
	\subparagraph{Actual Test output}
	Removed constraint and displayed list of constraints with constraint of interest removed
	
\paragraph{Requirement 5 - Update Constraint}
	\subparagraph{Expected Test output}
	Change constraint details to those specified in the field (size) and display list of constraints with the updated constraint included
	
	\subparagraph{Actual Test output}
	Changed constraint details to those specified in the field (size) and displayed list of constraints with the updated constraint included
		\subsubsection{Non-Functional Testing }
			
\paragraph{Security}
	The authorisation in terms of who can change and create calculation of status methods was never done. 
	\subparagraph{Example of problem}
		For example, if a person wanted to change the way a status is calculated, every member of a Buzz Space will be able to do it. Only the lecturers are supposed to be able to do this. 


	
	\subsection{Threads Comparison}
	According to the above tests both teams failed to integrate Threads with in the overall system, as all covered tests were failures and did not provide the expected functionality for the reasons given above.
	
\newpage 
\section{Status Integration}
The tests provided below show all cases in which the Status module was to make use of another Content module and the expected and actual results of each tests. We then explain which non-functional requirements have not been fulfilled in the overall Status module.
	\subsection{Status Team A}	
		\subsubsection{Functional Testing}
			
\paragraph{Requirement 1 - Upload Resources}
	\subparagraph{Expected Test output}
	Add the uploaded resource to the list of resources, and display the list
	
	\subparagraph{Actual Test output}
	Option to upload available, but not functioning (does not add resource to the list) 
		\begin{itemize}
			\item Because the team didn't integrate this, implemented the function
		\end{itemize}
		
\paragraph{Requirement 2 - Remove Resource}
	\subparagraph{Expected Test output}
	Remove resource and display list of resources with resource of interest removed
	
	\subparagraph{Actual Test output}
	Removed resource and displayed list of resources with resource of interest removed
	
\paragraph{Requirement 3 - Add Constraint}
	\subparagraph{Expected Test output}
	Add constraint with specified details and display list of constraints with the constraint of added 
	
	\subparagraph{Actual Test output}
	Added constraint with specified details and display list of constraints with the constraints of added included
	
\paragraph{Requirement 4 - Remove Constraint}
	\subparagraph{Expected Test output}
	Remove constraint and display list of constraints with constraint of interest removed
	
	\subparagraph{Actual Test output}
	Removed constraint and displayed list of constraints with constraint of interest removed
	
\paragraph{Requirement 5 - Update Constraint}
	\subparagraph{Expected Test output}
	Change constraint details to those specified in the field (size) and display list of constraints with the updated constraint included
	
	\subparagraph{Actual Test output}
	Changed constraint details to those specified in the field (size) and displayed list of constraints with the updated constraint included
		\subsubsection{Non-Functional Testing }
			
\paragraph{Security}
	The authorisation in terms of who can change and create calculation of status methods was never done. 
	\subparagraph{Example of problem}
		For example, if a person wanted to change the way a status is calculated, every member of a Buzz Space will be able to do it. Only the lecturers are supposed to be able to do this. 


	
	\subsection{Status Team B}	
		\subsubsection{Functional Testing}
			
\paragraph{Requirement 1 - Upload Resources}
	\subparagraph{Expected Test output}
	Add the uploaded resource to the list of resources, and display the list
	
	\subparagraph{Actual Test output}
	Option to upload available, but not functioning (does not add resource to the list) 
		\begin{itemize}
			\item Because the team didn't integrate this, implemented the function
		\end{itemize}
		
\paragraph{Requirement 2 - Remove Resource}
	\subparagraph{Expected Test output}
	Remove resource and display list of resources with resource of interest removed
	
	\subparagraph{Actual Test output}
	Removed resource and displayed list of resources with resource of interest removed
	
\paragraph{Requirement 3 - Add Constraint}
	\subparagraph{Expected Test output}
	Add constraint with specified details and display list of constraints with the constraint of added 
	
	\subparagraph{Actual Test output}
	Added constraint with specified details and display list of constraints with the constraints of added included
	
\paragraph{Requirement 4 - Remove Constraint}
	\subparagraph{Expected Test output}
	Remove constraint and display list of constraints with constraint of interest removed
	
	\subparagraph{Actual Test output}
	Removed constraint and displayed list of constraints with constraint of interest removed
	
\paragraph{Requirement 5 - Update Constraint}
	\subparagraph{Expected Test output}
	Change constraint details to those specified in the field (size) and display list of constraints with the updated constraint included
	
	\subparagraph{Actual Test output}
	Changed constraint details to those specified in the field (size) and displayed list of constraints with the updated constraint included
		\subsubsection{Non-Functional Testing }
			
\paragraph{Security}
	The authorisation in terms of who can change and create calculation of status methods was never done. 
	\subparagraph{Example of problem}
		For example, if a person wanted to change the way a status is calculated, every member of a Buzz Space will be able to do it. Only the lecturers are supposed to be able to do this. 


	
	\subsection{Status Comparison}
	Team A managed to implement 100 percent of the functions they had to do where as Team B only managed to implement a third of the  functionality. Overall however both tests failed in a sense because of the failure to run on Linux that Team A presented and the lack of functionality by Team B.
	
\newpage 
\section{Resources Integration}
The tests provided below show all cases in which the Resources module was to make use of another Content module and the expected and actual results of each tests. We then explain which non-functional requirements have not been fulfilled in the overall Resources module.
	\subsection{Resources Team A}	
		\subsubsection{Functional Testing}
			
\paragraph{Requirement 1 - Upload Resources}
	\subparagraph{Expected Test output}
	Add the uploaded resource to the list of resources, and display the list
	
	\subparagraph{Actual Test output}
	Option to upload available, but not functioning (does not add resource to the list) 
		\begin{itemize}
			\item Because the team didn't integrate this, implemented the function
		\end{itemize}
		
\paragraph{Requirement 2 - Remove Resource}
	\subparagraph{Expected Test output}
	Remove resource and display list of resources with resource of interest removed
	
	\subparagraph{Actual Test output}
	Removed resource and displayed list of resources with resource of interest removed
	
\paragraph{Requirement 3 - Add Constraint}
	\subparagraph{Expected Test output}
	Add constraint with specified details and display list of constraints with the constraint of added 
	
	\subparagraph{Actual Test output}
	Added constraint with specified details and display list of constraints with the constraints of added included
	
\paragraph{Requirement 4 - Remove Constraint}
	\subparagraph{Expected Test output}
	Remove constraint and display list of constraints with constraint of interest removed
	
	\subparagraph{Actual Test output}
	Removed constraint and displayed list of constraints with constraint of interest removed
	
\paragraph{Requirement 5 - Update Constraint}
	\subparagraph{Expected Test output}
	Change constraint details to those specified in the field (size) and display list of constraints with the updated constraint included
	
	\subparagraph{Actual Test output}
	Changed constraint details to those specified in the field (size) and displayed list of constraints with the updated constraint included
		\subsubsection{Non-Functional Testing }
			
\paragraph{Security}
	The authorisation in terms of who can change and create calculation of status methods was never done. 
	\subparagraph{Example of problem}
		For example, if a person wanted to change the way a status is calculated, every member of a Buzz Space will be able to do it. Only the lecturers are supposed to be able to do this. 


	
	\subsection{Resources Team B}	
		\subsubsection{Functional Testing}
			
\paragraph{Requirement 1 - Upload Resources}
	\subparagraph{Expected Test output}
	Add the uploaded resource to the list of resources, and display the list
	
	\subparagraph{Actual Test output}
	Option to upload available, but not functioning (does not add resource to the list) 
		\begin{itemize}
			\item Because the team didn't integrate this, implemented the function
		\end{itemize}
		
\paragraph{Requirement 2 - Remove Resource}
	\subparagraph{Expected Test output}
	Remove resource and display list of resources with resource of interest removed
	
	\subparagraph{Actual Test output}
	Removed resource and displayed list of resources with resource of interest removed
	
\paragraph{Requirement 3 - Add Constraint}
	\subparagraph{Expected Test output}
	Add constraint with specified details and display list of constraints with the constraint of added 
	
	\subparagraph{Actual Test output}
	Added constraint with specified details and display list of constraints with the constraints of added included
	
\paragraph{Requirement 4 - Remove Constraint}
	\subparagraph{Expected Test output}
	Remove constraint and display list of constraints with constraint of interest removed
	
	\subparagraph{Actual Test output}
	Removed constraint and displayed list of constraints with constraint of interest removed
	
\paragraph{Requirement 5 - Update Constraint}
	\subparagraph{Expected Test output}
	Change constraint details to those specified in the field (size) and display list of constraints with the updated constraint included
	
	\subparagraph{Actual Test output}
	Changed constraint details to those specified in the field (size) and displayed list of constraints with the updated constraint included
		\subsubsection{Non-Functional Testing }
			
\paragraph{Security}
	The authorisation in terms of who can change and create calculation of status methods was never done. 
	\subparagraph{Example of problem}
		For example, if a person wanted to change the way a status is calculated, every member of a Buzz Space will be able to do it. Only the lecturers are supposed to be able to do this. 


			
	
	\subsection{Resources Comparison}
	
	
\newpage 
\section{Reporting Integration}
The tests provided below show all cases in which the Reporting module was to make use of another Content module and the expected and actual results of each tests. We then explain which non-functional requirements have not been fulfilled in the overall Reporting module.
	\subsection{Reporting Team A}	
		\subsubsection{Functional Testing}
			
\paragraph{Requirement 1 - Upload Resources}
	\subparagraph{Expected Test output}
	Add the uploaded resource to the list of resources, and display the list
	
	\subparagraph{Actual Test output}
	Option to upload available, but not functioning (does not add resource to the list) 
		\begin{itemize}
			\item Because the team didn't integrate this, implemented the function
		\end{itemize}
		
\paragraph{Requirement 2 - Remove Resource}
	\subparagraph{Expected Test output}
	Remove resource and display list of resources with resource of interest removed
	
	\subparagraph{Actual Test output}
	Removed resource and displayed list of resources with resource of interest removed
	
\paragraph{Requirement 3 - Add Constraint}
	\subparagraph{Expected Test output}
	Add constraint with specified details and display list of constraints with the constraint of added 
	
	\subparagraph{Actual Test output}
	Added constraint with specified details and display list of constraints with the constraints of added included
	
\paragraph{Requirement 4 - Remove Constraint}
	\subparagraph{Expected Test output}
	Remove constraint and display list of constraints with constraint of interest removed
	
	\subparagraph{Actual Test output}
	Removed constraint and displayed list of constraints with constraint of interest removed
	
\paragraph{Requirement 5 - Update Constraint}
	\subparagraph{Expected Test output}
	Change constraint details to those specified in the field (size) and display list of constraints with the updated constraint included
	
	\subparagraph{Actual Test output}
	Changed constraint details to those specified in the field (size) and displayed list of constraints with the updated constraint included
		\subsubsection{Non-Functional Testing }
			
\paragraph{Security}
	The authorisation in terms of who can change and create calculation of status methods was never done. 
	\subparagraph{Example of problem}
		For example, if a person wanted to change the way a status is calculated, every member of a Buzz Space will be able to do it. Only the lecturers are supposed to be able to do this. 


	
	\subsection{Reporting Team B}	
		\subsubsection{Functional Testing}
			
\paragraph{Requirement 1 - Upload Resources}
	\subparagraph{Expected Test output}
	Add the uploaded resource to the list of resources, and display the list
	
	\subparagraph{Actual Test output}
	Option to upload available, but not functioning (does not add resource to the list) 
		\begin{itemize}
			\item Because the team didn't integrate this, implemented the function
		\end{itemize}
		
\paragraph{Requirement 2 - Remove Resource}
	\subparagraph{Expected Test output}
	Remove resource and display list of resources with resource of interest removed
	
	\subparagraph{Actual Test output}
	Removed resource and displayed list of resources with resource of interest removed
	
\paragraph{Requirement 3 - Add Constraint}
	\subparagraph{Expected Test output}
	Add constraint with specified details and display list of constraints with the constraint of added 
	
	\subparagraph{Actual Test output}
	Added constraint with specified details and display list of constraints with the constraints of added included
	
\paragraph{Requirement 4 - Remove Constraint}
	\subparagraph{Expected Test output}
	Remove constraint and display list of constraints with constraint of interest removed
	
	\subparagraph{Actual Test output}
	Removed constraint and displayed list of constraints with constraint of interest removed
	
\paragraph{Requirement 5 - Update Constraint}
	\subparagraph{Expected Test output}
	Change constraint details to those specified in the field (size) and display list of constraints with the updated constraint included
	
	\subparagraph{Actual Test output}
	Changed constraint details to those specified in the field (size) and displayed list of constraints with the updated constraint included
		\subsubsection{Non-Functional Testing }
			
\paragraph{Security}
	The authorisation in terms of who can change and create calculation of status methods was never done. 
	\subparagraph{Example of problem}
		For example, if a person wanted to change the way a status is calculated, every member of a Buzz Space will be able to do it. Only the lecturers are supposed to be able to do this. 


	
	\subsection{Reporting Comparison}

\end{document}
