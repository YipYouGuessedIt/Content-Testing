\paragraph{Architecture}
In accordance with the project specifications, the architectural requirements were specified as:
    \section{Access Channels}
    \paragraph{This section refers to the ways in are accessible for use}
        \begin{itemize}
            \item \subsection{Human Access Channels}
                \begin{itemize}
                    \item Accessible by human users through multiple platforms i.e. web browsers and mobile applications.
                    \item This is more towards the final finished product and will not be looked at in this section.
                \end{itemize}
                
            \item \subsection{System Access Channels}
                \begin{itemize}
                    \item Other services should be able to the service through a convenient access channel  supported across different technologies and frameworks.
                    \item The project section was implemented as a nodejs module and place within a github repository. This provides convenience of access as it can simply be installed as a node module.
                \end{itemize}
                
            \item \subsection{Integration Channels}
                \begin{itemize}
                    \item The system is able to access the CS LDAP server to restrive person details and class list; the system is able to access the CS MySQL database to access course/module information.
                    \item The system uses a node module databaseStuff which handles the connection and basic access to the CS LDAP server. They do not however, access the CS MySQL database.
                \end{itemize}
                
        \end{itemize}
    
    \section{Quality Requirements}
    \paragraph{This section looks st the manner in which the code is made and maintained}
        \begin{itemize}
            \item \subsection{Maintainability}
                \begin{itemize}
                    \item How flexible and extensible is the code and, can the system be easily maintained in future?
                    \item The developers implemented their system very simplistically and provided comments with explanations of their code. The comments provided contain expected parameters, expected results and, potential throws.
                \end{itemize}
                
            \item \subsection{Scalability}
                \begin{itemize}
                    \item The system should be easily scalable to large systems from the University of Pretoria to even larger systems.
                    \item Due to it's simplistic nature and implementation, the system could easily be scaled.
                \end{itemize}
                
            \item \subsection{Performance Requirements}
                \begin{itemize}
                    \item Reporting operations should take no more than 5 seconds while non-reporting operations should take less than (or equal to) 0.2 seconds.
                    \item This system does not work so we cannot test.
                \end{itemize}
                
            \item \subsection{Reliability and Availability}
                \begin{itemize}
                    \item Should support fail-safe over safety of components and deployment without a single point of failure.
                    \item The system itself completely failed as there were multiple errors.
                \end{itemize}
                
            \item \subsection{Security}
                \begin{itemize}
                    \item Check authentication against LDAP and flexible configurable authorization framework.
                    \item No authorization is handled.
                \end{itemize}
                
        \end{itemize}
